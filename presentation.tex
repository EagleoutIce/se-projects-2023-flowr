\documentclass[aspectratio=169,english,plain,c]{beamer}

\usepackage{lib/beamer-themes/dividing-lines/beamerthemedividing-lines}

\usepackage[T1]{fontenc}
\usepackage[utf8]{inputenc}
\usepackage{tikz}
\usepackage{fontawesome}
\usepackage{etoolbox}
\usetikzlibrary{backgrounds,graphs,arrows.meta}

\definecolor{BaseGray}{RGB}{66,66,66} % rgb(66,66,66)

\colorlet{SoftGray}{BaseGray!40}
\colorlet{BackGray}{BaseGray!5}
\colorlet{SoftTextGray}{BackGray!60!SoftGray}

\tikzset{
   FunctionDef/.style={
      draw=BaseGray,
      fill=BaseGray,
      minimum width=1.55cm,
      minimum height=1cm,
      text=white,
      font=\bfseries,
      text centered,
      inner sep=0pt,
      rounded corners=1mm,
      outer sep=2pt
   },
   Blob/.style={
      draw=SoftGray,
      fill=SoftGray,
      minimum size=4mm,
      text=white,
      circle,
      font=\bfseries,
      text centered,
      inner sep=0pt,
      outer sep=2pt
   },
   Def/.style={
      Blob,
      rectangle, rounded corners=1mm
   },
   ActiveBlob/.style={
      Blob,
      draw=BaseGray, fill=BaseGray
   },
   FunctionBack/.style={
      fill=BackGray,
      % draw=SoftGray,
      rounded corners=2mm,
      rectangle
   },
   Link/.style={
      draw=SoftGray,
      line width=1.5pt,
      line cap=round,
      line join=round,
      -%
   },
   FuncLink/.style={
      Link,
      draw=SoftGray,
      dotted
   },
   Cursor/.style={
      fill=BackGray,
      draw=BaseGray,
      line join=round,
      line cap=round
   },
   Hover-Over/.style={
      fill=BackGray,
      draw=SoftGray,
      opacity=.5,
      draw opacity=1,
      rounded corners=1mm,
      line join=round,
      line cap=round
   },
   Line-Of-Text/.style={
      fill=#1,
      draw=none,
      rounded corners=1.5pt,
      inner sep=1pt,
      minimum width=1cm,
      minimum height=6.5pt
   },
   Input-Base/.style={
      fill=BackGray,
      draw=SoftGray,
      rounded corners=1mm,
      inner sep=1pt,
      minimum width=2cm,
      minimum height=12pt
   },
   % code sub-styles
   A/.style={Line-Of-Text=SoftTextGray},
   B/.style={},
   C/.style={Line-Of-Text=SoftGray}
}

\usepackage[english]{babel}
\usepackage{microtype}

\usepackage{etoolbox,tabularx,array}
\makeatletter
\appto\input@path{{lib/color-palettes/}{lib/sopra-collection/sopra-listings}{lib/code-animation/}}

\usepackage{lib/color-palettes/color-palettes}
\colorlet{paletteA}{BaseGray}
\colorlet{paletteB}{BaseGray}
\colorlet{paletteC}{BaseGray}
\colorlet{paletteD}{BaseGray}
\usepackage[cpalette,encoding,defaultfont,fakeminted]{lib/sopra-collection/sopra-listings/sopra-listings}

\solLoadLanguage{R}
\solsetmintedstyle{plain}
\usepackage{lib/code-animation/code-animation}
\SetCodeAnimationFadeOutMixin{!10!btdl@color@background}
\CodeAnimationsWithFadeOut

\SolDefineStyles{%
 {keywordA: \color{sol@colors@lst@keywordA}\bfseries},%
 {keywordB: \color{sol@colors@lst@keywordB}\bfseries},%
 {keywordC: \color{sol@colors@lst@keywordC}\bfseries},%
 {keywordD: \color{sol@colors@lst@keywordC}\bfseries},%
}

\usepackage[bare]{lib/tikzpingus/tex/tikzpingus}

\def\DoTitlepage{}%
\def\DoEndpage{}%

\title{SE-Projekte im WS 23/24 --- flowR}
\subtitle{Statische Datenflusss-Analyse von R-Code}
\author{Florian Sihler}
\date{20.07.2023}

\newsavebox\HappyPingu
\savebox\HappyPingu{\tikz{\pingu[eyes wink,right wing wave,:mix-all=30!btdl@color@background]}}

\begin{document}

\begin{frame}
\begin{tikzpicture}[overlay,remember picture]
   \node[below left=3mm] (@flowr) at(current page.north east) {%
      \includegraphics[width=2.5cm]{data/flowR.pdf}
   };
   % fake opacity
   \fill[btdl@color@background,opacity=.5] (@flowr.north east) rectangle (@flowr.south west);
   \node[below right,xshift=5mm] (@) at(current page.west) {%
      \usebeamerfont{title}Titel%
   };
   \node[below right,font=\itshape\sbfamily,SoftGray] at (@.south west) {%
      im Wintersemester 2023/24%
   };
   \node[above left=2.5mm,scale=.2] (@) at(current page.south east) {\quad\usebox\HappyPingu};
   \node[above right=2.5mm,opacity=.55,scale=.2] at(current page.south west) {\includegraphics[height=\dimexpr0.7\ht\HappyPingu+0.7\dp\HappyPingu]{data/sp.pdf}};
\end{tikzpicture}
\end{frame}

\begin{frame}
\begin{tikzpicture}[overlay,remember picture]
   % padding comes from preview border
   \node[above left=0mm] at (current page.south east) {%
      \includegraphics[width=.65\paperwidth]{build/visualizer-demo-graphic.pdf}
   };
   \node[below right=5mm,font=\sbfamily\Large] (@title) at(current page.north west) {%
      Interaktive Visualisierung von Datenflussgraphen%
   };
   \node[below right,yshift=-5mm,text width=.33\paperwidth] at(@title.south west) {%
      \begin{itemize}
         \itemsep7.5pt
         \item Interessante
         \item Fakten
         \item In
         \item Bulletpoints
      \end{itemize}
   };
\end{tikzpicture}
\end{frame}

\begin{frame}
\begin{tikzpicture}[overlay,remember picture]
   \node[below right] at (current page.north west) {%
      \includegraphics[width=.65\paperwidth]{build/reconstruct-demo-graphic.pdf}
   };
   \node[above left=5mm,yshift=-1.5mm,font=\sbfamily\Large] (@title) at(current page.south east) {%
      Rekonstruktion von Programm-Ausschnitten%
   };
   \node[above left,yshift=5mm,text width=.55\paperwidth] at(@title.north east) {%
      \begin{itemize}
         \item Richtig
         \item Gute
         \item Bulletpoints mit vielen Informationen
         \item Super Duper
      \end{itemize}
   };
\end{tikzpicture}
\end{frame}
\end{document}